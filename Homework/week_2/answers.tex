\documentclass[a4paper]{scrartcl}
\usepackage{amsmath}
\usepackage{amsfonts}
\usepackage{amssymb}
\usepackage[dutch]{babel}
\usepackage{braket}
\usepackage{a4wide}
\usepackage{listings}


\title{Answers week 2}
\author{Sebastiaan Arendsen}

\newcommand{\I}{\mathtt{i}}
\newcommand{\E}{\mathrm{e}}
\newcommand{\integral}[4]{\int_{#1}^{#2}\!#3\,\mathrm{d}#4}
\newcommand{\ointegral}[4]{\oint_{#1}^{#2}\!#3\,\mathrm{d}#4}


\begin{document}
\maketitle

\section{Explain the difference between == and ===}
In javascript == will check if the two values are equal. The === operator will also check if it has the same type. Example:

\begin{lstlisting}
0 == 0: True
0 == False: True
0 === 0: True
0 === False: False
\end{lstlisting}

\section{Explain what a closure is}


\section{Explain what higher order functions are}
+

\end{document}