\documentclass[a4paper]{scrartcl}
%\usepackage{amsmath}
%\usepackage{amsfonts}
%\usepackage{amssymb}
%\usepackage[dutch]{babel}
%\usepackage{braket}
\usepackage{a4wide}
\usepackage{listings}
\usepackage{color}
\usepackage{inputenc}[utf8]

\definecolor{lightgray}{rgb}{.9,.9,.9}
\definecolor{darkgray}{rgb}{.4,.4,.4}
\definecolor{purple}{rgb}{0.65, 0.12, 0.82}

% http://lenaherrmann.net/2010/05/20/javascript-syntax-highlighting-in-the-latex-listings-package
\lstdefinelanguage{JavaScript}{
  keywords={typeof, new, true, false, catch, function, return, null, catch, switch, var, if, in, while, do, else, case, break},
  keywordstyle=\color{blue}\bfseries,
  ndkeywords={class, export, boolean, throw, implements, import, this},
  ndkeywordstyle=\color{darkgray}\bfseries,
  identifierstyle=\color{black},
  sensitive=false,
  comment=[l]{//},
  morecomment=[s]{/*}{*/},
  commentstyle=\color{purple}\ttfamily,
  stringstyle=\color{red}\ttfamily,
  morestring=[b]',
  morestring=[b]"
}

\lstset{
   language=JavaScript,
   backgroundcolor=\color{lightgray},
   extendedchars=true,
   basicstyle=\footnotesize\ttfamily,
   showstringspaces=false,
   showspaces=false,
   numbers=left,
   numberstyle=\footnotesize,
   numbersep=9pt,
   tabsize=2,
   breaklines=true,
   showtabs=false,
   captionpos=b
}

\title{Answers week 2}
\author{Sebastiaan Arendsen}

\newcommand{\I}{\mathtt{i}}
\newcommand{\E}{\mathrm{e}}
\newcommand{\integral}[4]{\int_{#1}^{#2}\!#3\,\mathrm{d}#4}
\newcommand{\ointegral}[4]{\oint_{#1}^{#2}\!#3\,\mathrm{d}#4}


\begin{document}
\maketitle

\section{Explain the difference between == and ===}
In javascript == will check if the two values are equal. The === operator will also check if it has the same type. Example:

\begin{lstlisting}
0 == 0 //True
0 == False //True
0 === 0 //True
0 === False //False
\end{lstlisting}
\section{Explain what a closure is}
Javascript allows functions to be defined within other functions. This makes them able have private variables. Example:

\begin{lstlisting}
function foo(input) {
    var bar = input;
    return function(baz) {
        return bar + baz
    }
}

var testFunc = foo(5);
console.log(testFunc(3)); // 5 + 3 = 8
\end{lstlisting}
In this example the function testFunc is able to reach a variable declared in its ``parent'' function.

\section{Explain what higher order functions are}
Higher order functions are functions that return different functions. Example:

\begin{lstlisting}
var foo = function(bar, input) {
    return function (input) {
        return input.split(bar);
    }
}
\end{lstlisting}
This function can be used to create a csv splitter, or any other sv splitter. Example:
\begin{lstlisting}
var csvSplitter = foo(','); // function to split a string on every ','
var tsvSplitter = foo('\t'); // function to split a string on every tab
\end{lstlisting}

\section{Explain what a query selector is and give an example line of JavaScript that uses a query selector.}
Query selectors are ways with which you you can select elements based on CSS selectors. Example:
\begin{lstlisting}
<div class='foo'><input name='number'></div>
\end{lstlisting}

\begin{lstlisting}
<script>
    var number = document.querySelector('div.foo input[name='number']'); 
    // selects the first input named number in div class foo
</script>
\end{lstlisting} 

\end{document}